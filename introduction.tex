\section{Introduction}

Une base de connaissances a pour but de regrouper les connaissances d'un domaine spécifique et de les rendre exploitables par un ordinateur. Elle est généralement composée d'une base de faits, qui représente un ensemble de connaissances considérées comme vraies, et d'une base de règles (aussi appelée ontologie) qui permettent de faire des inférences à partir des faits à l'aide d'un moteur d'inférence.
\par Prenons pour exemple une base de connaissances contenant une base de faits avec un seul fait : "Socrate est un humain". Et une base de règles avec une seule règle : "Si X est un humain, alors X est mortel". 
\par Il est possible d'inférer des faits nouveaux. Dans notre exemple, on peut inférer "Socrate est mortel" à partir du fait et de la règle existants. On ajoute ce fait à notre base de faits et on constate qu'on ne peut plus rien inférer de nouveau : on a saturé la base de faits. Ce que l'on vient de faire s'appelle du chaînage avant : on ajoute des nouveaux faits à la base de faits en les inférant à partir des faits existants et des règles, jusqu'à qu'on ne puisse plus rien inférer de nouveau. 
\par Le plus souvent, une base de faits ne contient que des constantes, mais on peut aussi vouloir y intégrer des variables quantifiées existentiellement pour représenter des individus dont on sait qu'ils existent mais dont on ne connaît pas le nom. Et on peut également permettre d'inférer des faits sur des individus qu'on ne connaît pas grâce à ce qu'on appelle une règle existentielle. Par exemple, on peut créer la règle "Si X est humain, alors il existe un Y tel que X a un parent Y".
\par Mais ce type de règle peut être à l'origine de ce qu'on appelle des redondances. Par exemple, ajoutons la règle "Si X est un humain, alors il existe un Y et un Z tels que X a un parent Y et X a un parent Z" à note base de règles. On va pouvoir inférer à partir de "Socrate est un humain", que "il existe un Y tel que Socrate a un parent Y" et "il existe un Z tel que Socrate a un parent Z". Mais ces deux faits sont redondants, on a deux fois l'information qu'il existe un individu qui est parent de Socrate.
\par Les redondances qui peuvent être produites par les règles existentielles, en plus d'ajouter des faits inutiles à la base de faits, peuvent être à l'origine d'une augmentation importante du temps d'exécution nécessaire pour saturer une base de faits. Les faits redondants qui ont été produits peuvent en effet entraîner de nouvelles applications de règles qui peuvent elles-mêmes créer de nouvelles redondances, et ainsi de suite.
\par L'objectif de ce projet de TER est de travailler à la suppression de deux types de redondances : les redondances créées lors de l'exécution d'un algorithme de chaînage avant (suppression dynamique) et les redondances dans les bases de règles (suppression statique).
\par Nous avons donc étudié puis implémenté plusieurs algorithmes de chaînage avant dans Graal\footnote{\url{https://graphik-team.github.io/graal/}}, une bibliothèque réalisée par l'équipe GraphiK\footnote{\url{https://www.lirmm.fr/recherche/equipes/graphik}}. Et nous avons étudié puis implémenté des algorithmes de suppression de redondances dans les bases de règles.

\par Avant d'entrer dans le vif du sujet, il est nécessaire de définir certaines notions essentielles à la compréhension du sujet (section \ref{sec:preliminaires}). Puis, on présentera des algorithmes de chaînage avant ayant différentes propriétés en terme de suppression de redondances et de vitesse de calcul (section \ref{sec:types_de_chases}) afin d'ensuite pouvoir les implémenter (section \ref{sec:conception_implementation}). Ces implémentations ont permis de réaliser des expérimentations pour tester les propriétés de ces algorithmes (section \ref{sec:experimentation}). Enfin, on abordera la suppression des redondances dans les bases de règles (section \ref{sec:regles}).

% \par Notre travail a consisté à utiliser la bibliothèque Graal\footnote{\url{https://graphik-team.github.io/graal/}}, réalisée par l'équipe GraphiK\footnote{\url{https://www.lirmm.fr/recherche/equipes/graphik}}, pour implémenter plusieurs types de chaînages avant (section \ref{sec:conception_implementation}). ) ayant différentes propriétés concernant les redondances qu'ils évitent de créer (section \ref{sec:types_de_chases}). Et également à étudier les redondances au sein des règles pour pouvoir créer une application prenant une base de règles en entrée et retournant une base de règles sans redondance (section \ref{sec:regles}) - toujours en utilisant Graal.
% \par Enfin, des expérimentations ont été réalisées afin de comparer les différents types de chaînages avant en terme de vitesse d'exécution et de capacité à éviter les redondances (section \ref{sec:experimentation}), ainsi que des expérimentations servant à tester l'application supprimant les redondances dans les règles.

%\par Pour commencer, on introduit plus formellement les notions de base concernant les bases de connaissances (section \ref{sec:preliminaires}), puis on détaille les différents types de chaînage avant  pour ensuite expliquer le travail de conception et implémentation que nous avons effectué (section \ref{sec:conception_implementation}) et les expérimentations réalisées (section \ref{sec:experimentation}). Enfin, on explique le travail effectué pour la suppression des redondances dans les règles .