\section{Introduction}

Une base de connaissances a pour but de regrouper les connaissances d'un domaine spécifique et de les rendre exploitables par un ordinateur. Elle est généralement composée d'une base de faits, qui représente un ensemble de connaissances considérées comme vraies, et d'une base de règles (aussi appelée ontologie) qui permettent de faire des inférences à partir des faits à l'aide d'un moteur d'inférence.
\par Prenons pour exemple une base de connaissances contenant une base de faits avec un seul fait : "Socrate est un humain". Et une base de règles avec une seule règle : "Si X est un humain, alors X est mortel". 
\par On peut faire ce qu'on appelle une "requête" : on demande au moteur d'inférence "existe-t-il un X tel que X est mortel ?". Dans ce cas là, le moteur d'inférence remonte les règles "en arrière" pour trouver la réponse (ici il répond "Socrate"). On appelle cela un "chaînage arrière".
\par On peut aussi inférer des faits. Dans notre exemple, on peut inférer "Socrate est mortel" à partir du fait et de la règle existants. On ajoute ce fait à notre base de faits et on constate qu'on ne peut plus rien inférer de nouveau : on a saturé la base de faits. Ce que l'on vient de faire s'appelle du chaînage avant : on ajoute de nouveaux faits à la base de faits en les inférant à partir des faits existants et des règles. 
\par Ici, notre base de faits ne contient que des constantes (Socrate est une constante), mais on peut aussi vouloir y intégrer des variables quantifiées existentiellement pour représenter des individus dont on sait qu'ils existent mais dont on ne connaît pas le nom. Par exemple, ajoutons "Y est un humain" à notre base de faits. Ainsi, on sait qu'il existe un Y qui est un humain. 
\par Mais ici, on vient de créer ce qu'on appelle une "redondance" : en effet, on sait déjà qu'il existe un Y qui est humain : Socrate. On peut donc enlever ce fait sans perdre aucune connaissance.
\par Il s'agit du sujet de ce projet TER : la suppression des redondances dans les bases de connaissances.
\par Notre travail a consisté à utiliser la bibliothèque Graal\footnote{\url{https://graphik-team.github.io/graal/}}, réalisée par l'équipe GraphiK\footnote{\url{https://www.lirmm.fr/recherche/equipes/graphik}}, pour implémenter plusieurs types de chaînages avant (section \ref{sec:conception_implementation}) ayant différentes propriétés concernant les redondances qu'ils évitent de créer (section \ref{sec:types_de_chases}). Et également à étudier les redondances au sein des règles pour pouvoir créer une application prenant une base de règles en entrée et retournant une base de règles sans redondance (section \ref{sec:regles}) - toujours en utilisant Graal.
\par Enfin, des expérimentations ont été réalisées afin de comparer les différents types de chaînages avant en terme de vitesse d'exécution et de capacité à éviter les redondances (section \ref{sec:experimentation}), ainsi que des expérimentations servant à tester l'application supprimant les redondances dans les règles.

%\par Pour commencer, on introduit plus formellement les notions de base concernant les bases de connaissances (section \ref{sec:preliminaires}), puis on détaille les différents types de chaînage avant  pour ensuite expliquer le travail de conception et implémentation que nous avons effectué (section \ref{sec:conception_implementation}) et les expérimentations réalisées (section \ref{sec:experimentation}). Enfin, on explique le travail effectué pour la suppression des redondances dans les règles .